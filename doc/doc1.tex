\documentclass[a4paper, 11pt]{ctexart}
\usepackage{srcltx,graphicx}
\usepackage{amsmath, amssymb, amsthm}
\usepackage{color}
\usepackage{lscape}
\usepackage{multirow}
\usepackage{cases}
\usepackage{enumerate}
\usepackage{float}
\usepackage[ruled,vlined]{algorithm2e}
\usepackage{bm}
\usepackage{psfrag}
\usepackage{mathrsfs}
\usepackage[hang]{subfigure}
\numberwithin{equation}{section}
\numberwithin{figure}{section}

\newtheorem{theorem}{定理}
\newtheorem{example}{例}
\newtheorem{lemma}{引理}
\newtheorem{definition}{定义}
\newtheorem{remark}{注}
\newtheorem{comment}{注}
\newtheorem{conjecture}{Conjecture}

\setlength{\oddsidemargin}{0cm}
\setlength{\evensidemargin}{0cm}
\setlength{\textwidth}{150mm}
\setlength{\textheight}{230mm}

\newcommand\note[2]{{{\bf #1}\color{red} [ {\it #2} ]}}
%\newcommand\note[2]{{ #1 }} % using this line in the formal version

\newcommand\bbC{\mathbb{C}}
\newcommand\bbR{\mathbb{R}}
\newcommand\bbN{\mathbb{N}}

\newcommand\diag{\mathrm{diag}}
\newcommand\tr{\mathrm{tr}}
\newcommand\dd{\mathrm{d}}
\renewcommand\div{\mathrm{div}}

\newcommand\be{\boldsymbol{e}}
\newcommand\bx{\boldsymbol{x}}
\newcommand\bF{\boldsymbol{F}}
\newcommand\bG{\boldsymbol{G}}
\newcommand\bof{\boldsymbol{f}}
\newcommand\bu{\boldsymbol{u}}
\newcommand\bn{\boldsymbol{n}}
\newcommand\bv{\boldsymbol{v}}
\newcommand\bV{\boldsymbol{V}}
\newcommand\bU{\boldsymbol{U}}
\newcommand\bA{\boldsymbol{A}}
\newcommand\bB{\boldsymbol{B}}
\newcommand\vecf{\boldsymbol{f}}
\newcommand\hatF{\hat{F}}
\newcommand\flux{\hat{F}_{j+1/2}}
\newcommand\diff{\,\mathrm{d}}
\newcommand\Norm[1]{\lvert\lvert#1\rvert\rvert}

\newcommand\pd[2]{\dfrac{\partial {#1}}{\partial {#2}}}
\newcommand\od[2]{\dfrac{\dd {#1}}{\dd {#2}}}
\newcommand\abs[1]{\lvert #1 \rvert}
\newcommand\norm[1]{\lvert\lvert #1 \rvert\rvert}
\newcommand\bt{\bar\theta}
\newcommand\bd[1]{\bold{#1}}
\newcommand\pro[2]{\langle{#1},{#2}\rangle}

\title{辐射流体方程组调研}
\author{段俊明\thanks{北京大学数学科学学院,科学与工程计算系,邮箱: {\tt duanjm@pku.edu.cn}} }

\begin{document}
\maketitle
%\tableofcontents %If the document is very long
\section{辐射流体方程组}
以下主要使用[Radiation Hydrodynamics, Castor, 2003]这本书中的记号和推导方式来
介绍辐射流体方程组.
记辐射强度为$I_\nu$,这里下标$\nu$表示单个频率下的辐射强度,
记辐射传播方向为$\bn$,在三维时为
\begin{equation}
  \bn=\sin\theta\cos\phi\be_x+ \sin\theta\sin\phi\be_y+ \cos\theta\be_z,
\end{equation}
在三维情形下,$\bn$在一个单位球面上,该单位球面的面积元为$\dd\Omega$,
\begin{equation}
  \dd\Omega=\sin\theta\dd\theta\dd\phi,
\end{equation}
角度$\theta$在$[0,\pi]$区间内,角度$\phi$在$[0,2\pi]$区间内,需要注意的是
这里$\dd\Omega$使用球坐标的形式表达,并不意味着它与实际空间有关,辐射传播方向的角度空间与
实际空间是独立的.
不考虑源项带来的辐射强度增加或减少时,~$I_\nu$随时间和空间的变化满足辐射输运方程
\begin{equation}
  \dfrac{1}{c}\pd{I_\nu}{t}+\bn\cdot\nabla I_\nu=0,
\end{equation}
$\nabla$是空间的梯度算子.
现在考虑吸收和发射.经验规律表明,~$I_\nu$由于吸收导致的衰减与自身成正比,比例系数与
$I_\nu$无关,正比于平板厚度(若厚度不是特别大),可以认为穿过的平板厚度为$c\tau$,
$\tau$为时间元,
\begin{equation}
  \Delta I_\nu=-k_\nu c\tau I_\nu,
\end{equation}
称$k_\nu$为吸收系数,并定义光学厚度$\int k_\nu\dd l$.
发射会使得辐射强度增加,只依赖于平板厚度,
\begin{equation}
  \Delta I_\nu=+j_\nu c\tau,
\end{equation}
于是一般的辐射输运方程可以写成
\begin{equation}
  \dfrac{1}{c}\pd{I_\nu}{t}+\bn\cdot\nabla I_\nu=j_\nu-k_\nu I_\nu,
  \label{eq:RTE}
\end{equation}
在该形式下散射项已经包含在了吸收和发射项里面.

考虑$I_\nu$对于角度$\bn$的零阶矩,一阶矩,二阶矩,
\begin{align}
  E_\nu&=\dfrac{1}{c}\int_{4\pi}I_\nu\dd\Omega,\\
  F_\nu&=\int_{4\pi}\bn I_\nu\dd\Omega,\\
  P_\nu&=\dfrac{1}{c}\int_{4\pi}\bn\bn I_\nu\dd\Omega,
\end{align}
分别称为辐射能量密度,辐射通量,辐射压力,
若再将它们对于所有频率积分,得到总的辐射能量密度,辐射通量,辐射压力,
\begin{align}
  E_r&=\int_0^\infty E_\nu\dd\nu,\\
  F_r&=\int_0^\infty F_\nu\dd\nu,\\
  P_r&=\int_0^\infty P_\nu\dd\nu,
\end{align}
下标$r$表示与辐射相关.

对方程\eqref{eq:RTE}分别乘以$1,\bn$并在球面上和所有频率上积分,
\begin{align}
  \pd{E_r}{t}+\nabla\cdot F_r&=
  \int_0^\infty\int_{4\pi}(j_\nu-k_\nu I_\nu)\dd\Omega\dd\nu\triangleq S_E,\\
  \dfrac{1}{c^2}\pd{F_r}{t}+\nabla\cdot P_r&=
  \dfrac{1}{c}\int_0^\infty\int_{4\pi}\bn(j_\nu-k_\nu I_\nu)\dd\Omega\dd\nu\triangleq S_F,
\end{align}
$S_E,S_F$分别为辐射与物质交换的能量和动量.
现在考虑辐射与欧拉方程组的耦合,辐射与物质交换的能量和动量以源项的形式作用在欧拉
方程组上,即
\begin{align}
  \pd{\rho}{t}+\nabla\cdot(\rho\bu)&=0,\\
  \pd{(\rho\bu)}{t}+\nabla\cdot(\rho\bu\bu)+\nabla p&=-S_F,\\
  \pd{(\rho E)}{t}+\nabla\cdot[(\rho E+p)\bu]&=-S_E,
\end{align}
将辐射与物质的动量方程和能量方程相加,得到动量和能量的强守恒形式,
\begin{align}
\pd{}{t}(\rho\bu+\dfrac{F_r}{c^2})+\nabla\cdot(\rho\bu\bu+P_r)+\nabla p&=0,\\
  \pd{}{t}(\rho E+E_r)+\nabla\cdot[(\rho E+p)\bu+F_r]&=0.
\end{align}

选取适当的物理尺度进行无量纲化,得到
\begin{equation}
  (\dfrac{1}{\mathscr C}\pd{}{t}+\bn\cdot\nabla)I(\nu,\bn)=S(\nu,\bn),
\end{equation}
其中,~$\mathscr C=\dfrac{c}{a_\infty}$,~$c$是光速,~$a_\infty$是物质声速,
\begin{equation}
  S(\nu,\bn)={\mathscr L}\dfrac{\nu_0}{\nu}[(\dfrac{\nu}{\nu_0})^3
  \sigma_a(\nu_0)B(\nu_0,T)-\sigma_t(\nu_0)I(\nu,\bn)]+\eta^S(\nu,\bn).
\end{equation}
动量和能量方程变为
\begin{align}
  \pd{}{t}(\rho\bu+\dfrac{\mathscr P}{\mathscr
  C}F_r)+\nabla\cdot(\rho\bu\bu+{\mathscr P}P_r)+\nabla p&=0,\\
  \pd{}{t}(\rho E+{\mathscr P}E_r)+\nabla\cdot[(\rho E+p)\bu+{\mathscr {PC}}F_r]&=0.
\end{align}

\section{low energy density regime}
[On Consistent Time-Integration Methods for Radiation Hydrodynamics in
the Equilibrium Diffusion Limit: Low-Energy-Density Regime, J. W. Bates et al, 2001, JCP]
中给出了灰体近似和扩散极限下,且辐射场和物质保持LTE,非相对论情形下的辐射流体方程组,
\begin{align}
  \pd{\rho}{t}+\nabla\cdot(\rho\bu)&=0,\\
  \pd{}{t}(\rho\bu)+\nabla\cdot(\rho\bu\bu)+\nabla (p+P_\nu)&=0,\\
  \pd{}{t}(E+E_\nu)+\nabla\cdot[(E+E_\nu+p+p_\nu)\bu]&=\nabla\cdot(\kappa\nabla T),
\end{align}
其中,~$p_\nu=E_\nu/3=\dfrac43\sigma T^4/c$,~$p=R\rho T=(\gamma-1)\rho e$,
在low energy density regime中,~$p_\nu,E_\nu$相对于$p,E$可以忽略不计.
考虑一维球对称情形,该方程组化为
\begin{align}
  \pd{\rho}{t}+\dfrac{1}{r^2}\pd{}{r}(r^2\rho u)&=0,\\
  \pd{}{t}(\rho u)+\dfrac{1}{r^2}\pd{}{r}(r^2\rho u^2)+\pd{p}{r}&=0,\\
  \pd{E}{t}+\dfrac{1}{r^2}\pd{}{r}[r^2(E+p)u]&=
  \dfrac{1}{r^2}\pd{}{r}(r^2\kappa \pd{T}{r}).
\end{align}
定义比热容$C_v=\dfrac{R}{\gamma-1}$,则$E=C_v\rho T+\dfrac12\rho u^2$,
这里假定$\kappa=\kappa_0\rho^a T^b$,在模型中$b=2.5,6.5$,因此$\kappa$可能很大,
这一项需要隐式处理.

Kadioglu, Knoll在2001年JCP的文章中使用一种IMEX方法求解该方程组,即显式Godunov格式
求解流体方程组,隐式格式求解热传导项,隐式部分使用Newton-Krylov迭代法求解非线性方
程组,时间方向使用predictor-corrector两层格式,显式求解部分与隐式求解部分分离,不能
达到二阶精度.在2010年JCP的文章中,他们将显式求解部分放到了隐式迭代求解中,即每一次
非线性方程组的迭代都要显式求解一次流体方程组,该格式的计算量较大,但达到了二阶精度
.之后在2010年JCP的文章中,他们将该方法推广到了high energy density的方程组中.
Zhang Xiangxiong在2017年JCP的文章中,提出了一种显式的高精度保正DG方法求解NS方程组
,时间步长为$O(Re h^2)$,该方法适用于高雷诺数的情形,由于这里热传导系数较大,该方法
并不适用.

Newton迭代法需要一阶导数信息,在高维时需要生成Jacobi矩阵并求逆,这部分计算量过大,
因此这里考虑使用Krylov子空间方法来处理一步Newton迭代,~$J(T^k)\delta T^k=-F(T^k)$,
例如GMRES方法只需要矩阵向量乘积,这一部分是容易得到的.

下面给出Kadioglu和Knoll在2010年JCP的文章中提出的算法.
将流体方程组写为
\begin{equation}
  \pd{\bU}{t}+\pd{(A\bF)}{V}+\pd{\bG}{r}=0,
\end{equation}
其中$\bU=(\rho,\rho u,E)$,~$\bF(\bU)=(\rho u,\rho
u^2,u(E+p))$,~$\bG(\bU)=(0,p,0)$,~$V=\frac43\pi r^3$,~$A=4\pi r^2$.
使用二阶RK求解,
\begin{align}
  \bU^1&=\bU^n+\Delta tL(\bU^n),\\
  \bU^{n+1}&=\frac12\bU^n+\frac12\bU^1+\frac12\Delta tL(\bU^{1,*}),
\end{align}
使用LLF通量,
\begin{align}
  \bU^1_i&=\bU^n_i-\dfrac{\Delta t}{\Delta V_i}(A_{i+\frac12}\bF^n_{i+\frac12}
  -A_{i-\frac12}\bF^n_{i-\frac12})-\dfrac{\Delta t}{\Delta
  r}(\bG^n_{i+\frac12}-\bG^n_{i+\frac12}),\\
  \bF_{i+\frac12}&=\dfrac{\bF(\bU^R_{i+\frac12})+\bF(\bU^L_{i+\frac12})}{2}
  -\alpha_{i+\frac12}\dfrac{\bU^R_{i+\frac12}-\bU^L_{i+\frac12}}{2},
\end{align}
$\bU^R_{i+\frac12}, \bU^L_{i+\frac12}$使用线性重构和minmod限制器得到.
与常见的格式唯一不同之处是,
\begin{equation}
  E^{1,*}=C_v\rho^1 T^k+\frac12\rho^1(u^1)^2,
\end{equation}
使用了隐式第k次迭代得到的温度$T^k$.


整体算法如下,
\begin{itemize}
  \item 在$t^n$时,使用Newton迭代求解隐式求解热传导项,假设当前是第k次迭代,~$T^k$已知.
  \item 使用二阶RK和LLF格式求解一次流体方程组,得到$(\rho^{n+1},(\rho u)^{n+1},E^*)$,
  \item 形成非线性方程组的残量,
    \begin{align}
      &Res=\dfrac{C_v\rho^{n+1}T^k+\rho^{n+1}(u^{n+1})^2/2-E^*}{\Delta
      t}-\dfrac{RHS(\rho^{n+1},T^k)+RHS(\rho^{n},T^n)}{2},\\
      &RHS(\rho_i,T_i)=\dfrac{A_{i+\frac12}\kappa_{i+\frac12}(T_{i+1}-T_i)/\Delta r}{\Delta V_i}-
      -\dfrac{A_{i-\frac12}\kappa_{i-\frac12}(T_{i}-T_{i-1})/\Delta r}{\Delta V_i},
    \end{align}
    计算$\delta^k T,k=k+1$,直到$\norm{\bF(T^k)}<\varepsilon$.
\end{itemize}

现在考虑隐式求解非线性热传导项的保正,若$\kappa$为一个正的常系数,形成的线性方程组
的解自然是正的,关键是非线性$\kappa$时如何处理.

\end{document}

